\intro

Оптимизация приложения представляет собой процесс, в результате которого достигается приемлемое быстродействие, которое и ожидает пользователь~\cite{effProg}. Обычно, пользователи не в восторге от медленно выполняющихся приложений и, скоре всего, перестают ими пользоваться. Пользователь предпочтёт приложение, которое занимает мало места на жестком диске, отзывчива и потребляет минимальное количество системных ресурсов.

Для оптимизации было выбрано существующее Electron-приложение, представляющее собой текстовый редактор с возможностью проверки орфографии, но работающее медленно и потребляющее большой объем оперативной памяти. Целью настоящей практики является изучения способов оптимизации Electron-приложений для создания эффективного приложения.

Для достижения цели были поставлены следующие задачи:

\begin{itemize}
  \item анализ структуры существующего Electron-приложения;
  \item выявление критических мест приложения, понижающих его производительность;
  \item изучение способов работы с библиотекой React и фреймворком Electron;
  \item разработка эффективного Electron-приложения.
\end{itemize}