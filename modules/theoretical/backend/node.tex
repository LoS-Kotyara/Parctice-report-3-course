\subsubsection{Node.js}

Node.js представляет собой среду выполнения Javascript-кода, вне браузера~\cite{node}. Node.js основан на движке Javascript V8, который позволяет транслировать JS-код в машинный код. Прежде всего, Node.js предназначен для создания серверных и настольных приложений.

Одним из преимуществ Node.js является то, что если разработчик владеет навыками написания frontend Javascript кода, то он без проблем сможет разработать backend код, и благодаря этому он является одним из популярных языков написания серверных приложений.

Так же одним из достоинств языка является наличие механизма асинхронного ввода-вывода, а именно использование неблокирующих операций ввода-вывода~\cite{node}. Это значит, что главный поток не будет блокироваться операциями ввода-вывода и сервер будет продолжать обрабатывать запросы. Это возможно из-за того, что в V8, а, следовательно, Node.js, используется цикл событий, который ожидает прибытия и проводит рассылку событий и выполняет операции только когда произошло определённое событие.

