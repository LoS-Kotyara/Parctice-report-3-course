\subsubsection{HTML}

HTML (HyperText Markup Language - ``Язык гипертекстовой разметки'') - это стандартизированный язык разметки документов, определяющий содержание и структуру веб-контента.

Веб-браузер может интерпретировать описанный с помощью HTML документ и отобразить его структуру на экране пользователя.

При загрузке HTML браузером, создаётся объектная модель документа~-~DOM. DOM - это стандартная объектная модель программного интерфейса HTML, которая определяет:

\begin{itemize}
  \item Элементы HTML как объекты;
  \item Свойства всех HTML элементов;
  \item Методы доступа ко всем HTML элементам;
  \item События для всех HTML элементов.
\end{itemize}

Другими словами, DOM - это стандарт того, как можно получить, изменить, добавить или удалить HTML элементы.

Благодаря DOM можно описать все элементы как набор утверждений и формул, изменение которых ведёт к автоматическому перерасчёту всех зависимостей, и, таким образом, сделать сайт реактивным, т.е. способным обновляться при изменении внутреннего состояния, с помощью, например, JS.

Реактивность реализована во многих frontend-фреймворках и библиотеках. Я выбрал React, так как он даёт более высокий уровень гибкости при разработке.