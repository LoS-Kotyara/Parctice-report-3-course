\subsection{Разработка визуального интерфейса программы}

Визуальный интерфейс разрабатывался с использованием библиотеки React. Для подключения React к странице приложения, необходимо создать корневой элемент приложения (с id равным ``root''), к которому с помощью тега ``script'' будет подключаться библиотека и сгенерированное React-приложение. В листинге~\ref{lst:reactHTML} приведён код HTML-страницы, к которой производится подключение; код React-приложения приведен в листингах~\ref{lst:reacMain},~\ref{lst:reactSub}.

% \newpage

\lstinputlisting[style=ES6, caption={HTML-документ}, label = {lst:reactHTML}]{assets/listings/practical/reactHTML.html}

\lstinputlisting[style=ES6, caption={Главный файл React-приложения}, label = {lst:reacMain}]{assets/listings/practical/index.tsx}
